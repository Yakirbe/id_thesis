% Gray Code in 4-cube
% Author: Yury Chebiryak <http://yury.chebiryak.name/index.html>
\documentclass{article}
\usepackage{tikz}
%%%<
\usepackage{verbatim}
\usepackage[active,tightpage]{preview}
\PreviewEnvironment{tikzpicture}
\setlength\PreviewBorder{5pt}%
%%%>

\begin{comment}
:Title: Gray Code in 4-cube

Depicts a Gray code traversing all the nodes of a 4-dimensional hypercube. See `<http://yury.chebiryak.name/hypercubes.htm>`_ for more details.

\end{comment}
\begin{document}
	\pagestyle{empty}
	blablablabalbalbalba
	\begin{figure}[bt]
		\centering
		\scalebox{0.6}
		{
			\begin{tikzpicture}[scale=5]
			\tikzstyle{vertex}=[circle,minimum size=20pt,inner sep=0pt]
			\tikzstyle{selected vertex} = [vertex, fill=red!24]
			\tikzstyle{selected edge} = [draw,line width=5pt,-,red!50]
			\tikzstyle{edge} = [draw,thick,-,black]
			\node[vertex] (v0) at (0,0) {$0000$};
			\node[vertex] (v1) at (0,1) {$0001$};
			\node[vertex] (v2) at (1,0) {$0010$};
			\node[vertex] (v3) at (1,1) {$0011$};
			\node[vertex] (v4) at (0.23, 0.4) {$0100$};
			\node[vertex] (v5) at (0.23,1.4) {$0101$};
			\node[vertex] (v6) at (1.23,0.4) {$0110$};
			\node[vertex] (v7) at (1.23,1.4) {$0111$};
			\node[vertex] (v8) at (-1,-1) {$1000$};
			\node[vertex] (v9) at (-1,2) {$1001$};
			\node[vertex] (v13) at (-0.66,2.7) {$1101$};
			\node[vertex] (v12) at (-0.66,-0.3) {$1100$};
			\node[vertex] (v10) at (2,-1) {$1010$};
			\node[vertex] (v14) at (2.34,-0.3) {$1110$};
			\node[vertex] (v11) at (2,2) {$1011$};
			\node[vertex] (v15) at (2.34,2.7) {$1111$};
			\draw[edge] (v0) -- (v1) -- (v3) -- (v2) -- (v0);
			\draw[edge] (v0) -- (v4) -- (v5) -- (v1) -- (v0);
			\draw[edge] (v2) -- (v6) -- (v7) -- (v3) -- (v2);
			\draw[edge] (v4) -- (v6) -- (v7) -- (v5) -- (v4);
			\draw[edge] (v8) -- (v9) -- (v13) -- (v12) -- (v8);
			\draw[edge] (v0) -- (v4) -- (v12) -- (v8) -- (v0);
			\draw[edge] (v1) -- (v9) -- (v13) -- (v5) -- (v1);
			\draw[edge] (v2) -- (v10) -- (v14) -- (v6) -- (v2);
			\draw[edge] (v8) -- (v10) -- (v14) -- (v12) -- (v8);
			\draw[edge] (v3) -- (v11) -- (v15) -- (v7) -- (v3);
			\draw[edge] (v10) -- (v11) -- (v15) -- (v14) -- (v10);
			\draw[edge] (v9) -- (v11) -- (v15) -- (v13) -- (v9);
			\draw[selected edge] (v0) -- (v2);
			\draw[selected edge] (v2) -- (v6);
			\draw[selected edge] (v6) -- (v4);
			\draw[selected edge] (v4) -- (v5);
			\draw[selected edge] (v5) -- (v13);
			\draw[selected edge] (v13) -- (v12);
			\draw[selected edge] (v12) -- (v14);
			\draw[selected edge] (v14) -- (v15);
			\draw[selected edge] (v15) -- (v7);
			\draw[selected edge] (v7) -- (v3);
			\draw[selected edge] (v3) -- (v1);
			\draw[selected edge] (v1) -- (v9);
			\draw[selected edge] (v9) -- (v11);
			\draw[selected edge] (v11) -- (v10);
			\draw[selected edge] (v10) -- (v8);
			\draw[selected edge] (v8) -- (v0);
			\end{tikzpicture}
		}
		\caption{Complete Gray Code}
	\end{figure}

	\newcommand{\MyNewVariable}{(0.3 , 0.8)}
	\begin{figure}[bt]
		\centering
		\scalebox{0.6}
		{
			\begin{tikzpicture}[scale=5]
			\tikzstyle{vertex}=[circle,minimum size=20pt,inner sep=0pt]
			\tikzstyle{selected vertex} = [vertex, fill=red!6]
			\tikzstyle{selected edge} = [draw,line width=5pt,-,blue!50]
			\tikzstyle{edge} = [draw,thick,-,black]
			
			\node[vertex] (v00) at (-3,-3) {$(-3,-3)$};
			\node[vertex] (v01) at (-3,-2) {$(-3,-2)$};
			\node[vertex] (v02) at (-3,-1) {$(-3,-1)$};
			\node[vertex] (v03) at (-3,0) {$(-3,0)$};
			\node[vertex] (v04) at (-3,1) {$(-3,1)$};
			\node[vertex] (v05) at (-3,2) {$(-3,2)$};
			\node[vertex] (v06) at (-3,3) {$(-3,3)$};
			\node[vertex] (v10) at (-2,-3) {$(-2,-3)$};
			\node[vertex] (v11) at (-2,-2) {$(-2,-2)$};
			\node[vertex] (v12) at (-2,-1) {$(-2,-1)$};
			\node[vertex] (v13) at (-2,0) {$(-2,0)$};
			\node[vertex] (v14) at (-2,1) {$(-2,1)$};
			\node[vertex] (v15) at (-2,2) {$(-2,2)$};
			\node[vertex] (v16) at (-2,3) {$(-2,3)$};
			\node[vertex] (v20) at (-1,-3) {$(-1,-3)$};
			\node[vertex] (v21) at (-1,-2) {$(-1,-2)$};
			\node[vertex] (v22) at (-1,-1) {$(-1,-1)$};
			\node[vertex] (v23) at (-1,0) {$(-1,0)$};
			\node[vertex] (v24) at (-1,1) {$(-1,1)$};
			\node[vertex] (v25) at (-1,2) {$(-1,2)$};
			\node[vertex] (v26) at (-1,3) {$(-1,3)$};
			\node[vertex] (v30) at (0,-3) {$(0,-3)$};
			\node[vertex] (v31) at (0,-2) {$(0,-2)$};
			\node[vertex] (v32) at (0,-1) {$(0,-1)$};
			\node[vertex] (v33) at (0,0) {$(0,0)$};
			\node[vertex] (v34) at (0,1) {$(0,1)$};
			\node[vertex] (v35) at (0,2) {$(0,2)$};
			\node[vertex] (v36) at (0,3) {$(0,3)$};
			\node[vertex] (v40) at (1,-3) {$(1,-3)$};
			\node[vertex] (v41) at (1,-2) {$(1,-2)$};
			\node[vertex] (v42) at (1,-1) {$(1,-1)$};
			\node[vertex] (v43) at (1,0) {$(1,0)$};
			\node[vertex] (v44) at (1,1) {$(1,1)$};
			\node[vertex] (v45) at (1,2) {$(1,2)$};
			\node[vertex] (v46) at (1,3) {$(1,3)$};
			\node[vertex] (v50) at (2,-3) {$(2,-3)$};
			\node[vertex] (v51) at (2,-2) {$(2,-2)$};
			\node[vertex] (v52) at (2,-1) {$(2,-1)$};
			\node[vertex] (v53) at (2,0) {$(2,0)$};
			\node[vertex] (v54) at (2,1) {$(2,1)$};
			\node[vertex] (v55) at (2,2) {$(2,2)$};
			\node[vertex] (v56) at (2,3) {$(2,3)$};
			\node[vertex] (v60) at \MyNewVariable {};
		
			\draw[edge] (v00) -- (v01) -- (v02) -- (v03) -- (v04) -- (v05) -- (v06);
			\draw[edge] (v10) -- (v11) -- (v12) -- (v13) -- (v14) -- (v15) -- (v16);
			\draw[edge] (v20) -- (v21) -- (v22) -- (v23) -- (v24) -- (v25) -- (v26);
			\draw[edge] (v30) -- (v31) -- (v32) -- (v33) -- (v34) -- (v35) -- (v36);
			\draw[edge] (v40) -- (v41) -- (v42) -- (v43) -- (v44) -- (v45) -- (v46);
			\draw[edge] (v50) -- (v51) -- (v52) -- (v53) -- (v54) -- (v55) -- (v56);
			\draw[edge] (v00) -- (v10) -- (v20) -- (v30) -- (v40) -- (v50);
			\draw[edge] (v01) -- (v11) -- (v21) -- (v31) -- (v41) -- (v51);
			\draw[edge] (v02) -- (v12) -- (v22) -- (v32) -- (v42) -- (v52);
			\draw[edge] (v03) -- (v13) -- (v23) -- (v33) -- (v43) -- (v53);
			\draw[edge] (v04) -- (v14) -- (v24) -- (v34) -- (v44) -- (v54);
			\draw[edge] (v05) -- (v15) -- (v25) -- (v35) -- (v45) -- (v55);
			\draw[edge] (v06) -- (v16) -- (v26) -- (v36) -- (v46) -- (v56);
			
			\fill \MyNewVariable circle(1.3pt);
			\draw[edge] (v60) -- (v34);
			\draw[edge] (v60) -- (v44);
			\draw[edge] (v60) -- (v33);
			\draw[edge] (v44) -- (v33);

			 \fill[red] (0,0) -- (1,1) -- \MyNewVariable -- cycle;
			 \fill[green] (0,1) -- (1,1) -- \MyNewVariable -- cycle;
			 \fill[blue] (0,0) -- (0,1) -- \MyNewVariable -- cycle;
			\fill \MyNewVariable circle(1.3pt);
			
			
			
			\end{tikzpicture}
			
			\begin{tikzpicture}
			\draw (0,0) node[anchor=north, font = \huge]{$\overrightarrow{x}_3$}
			-- (10,0) node[anchor=north, font = \huge]{$\overrightarrow{x}_4$}
			-- (5,10) node[anchor=south, font = \huge]{$\overrightarrow{x}_2$}
			-- cycle;
			\draw (5,3.33) node[anchor=north, font = \huge]{$\overrightarrow{x}_1$};
			

			
			\draw[teal](2.7,5.5) node[left=7pt,fill=none, font = \huge]{$2$};
			\draw[teal](5.4,0.4) node[left=7pt,fill=none, font = \huge]{$2$};
			\draw[teal](8,5.5) node[left=7pt,fill=none , font = \huge]{$2$};
			
			\draw[teal](2.7,2) node[left=7pt,fill=none, font = \huge]{$1$};
			\draw[teal](7.5,2.4) node[left=7pt,fill=none, font = \huge]{$1$};
			\draw[teal](5.8,5.5) node[left=7pt,fill=none, font = \huge]{$1$};
			
			\draw (0,0) -- (5,3.33);
			\draw (5,10) -- (5,3.33);
			\draw (10,0) -- (5,3.33);
			\fill (5,3.33) circle(3pt);
			\fill (0,0) circle(3pt);
			\fill (10,0) circle(3pt);
			\fill (5,10) circle(3pt);
			\end{tikzpicture}
		}
		\caption{Complete Gray Code}
	\end{figure}








\end{document}