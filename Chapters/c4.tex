
\chapter{Interpolated Discretized Single objects Embedding} % Main chapter title

\label{Chapter4} % For referencing the chapter elsewhere, use \ref{Chapter1} 

Data objects may describe an abstracted format of any data type exists, such images, video, audio, text etc.
\textbf{Single} objects embedding may assist in classification and regression tasks, by emphasizing differences among data samples, without spending lots of time and memory.
\\
In this section we describe how we perform our ID method on such single vectors domain.



\section{Discretization}

We begin by dataset discretization, which is equivalent in all the use cases displayed in our method.
Dataset is being discretized by performing clustering on its dimensions and find C centers for discretization. This step should maintain the dataset extreme values within the extreme values of the C vector (so in the interpolation phase there will be valid values for all data elements \ref{fig:4.1}.


\begin{figure} 
\centering
		
		\begin{pmatrix} v_1 ,  v_2 , \dots , \dots , v_{n-1}, v_n} \end{pmatrix}\\
		
		\begin{pmatrix}
			m_{1,1}&   \dots&   \dots&   \dots& m_{1,n}\\
			\vdots& \ddots &        &        & \vdots \\
			\vdots&        & \ddots &        & \vdots  \\
			\vdots&        &        & \ddots & \vdots  \\
			m_{C-1,1} &        &        &        & m_{C-1,n}  \\
			m_{C,1}  & \dots  & \dots  & \dots  & m_{C,n}
		\end{pmatrix}
		\caption[discretization matrix]
		{discretization points set $W$ as built from a single dimensional dataset. this set is vectors' length may vary among dimensions}
\label{fig:4.1}		
\end{figure}


\section{Interpolation}

Interpolation of a given dataset, after extracting its discretization centers, is performed by the following sequence:

\begin{itemize}
\item Find bounding hypercube
\item Find bounding simplex
\item Find sample’s correlated coefficients per simplex vertices as described at \ref{hypercubes}
\item Convert coef. vector to  normalized format. $v_i \in [0,1]$  by dividing with the volume of the simplex
\end{itemize}

\section{Assigning}

Here we proceed embedding process by assigning the normalized coefficients vector to their indices in the embedded vector.
In the 1d scenario, the embedded vector would be sized as centers number per dimension - $C$, powered by dimension length - $n$



\begin{algorithm}
		\caption{Embedding Method for ID N-Dimensional single vectors dataset}
		\begin{algorithmic}
 
		
		\REQUIRE $L$ sized, vectorized n-dimensional dataset
		\REQUIRE set of centers per dimension - $C$
		\ENSURE $\overrightarrow{\phi}$: $L$ sized set, embedded, sparse vectors\\
		
		\STATE \textbf{Find centers vectors}
		\STATE V shall be a set of centers - vectors
		\FORALL{dim in $n$}  
		\STATE $V_{dim} \leftarrow centers \quad vector \quad per \quad dim$
		\ENDFOR
		
		\STATE \textbf{Find embedded coefficients for all dataset}
		\STATE $\overrightarrow{\phi} = C^{n}$ length empty $\overrightarrow{\phi}$ embedded vectors
		
		\FORALL{$\overrightarrow{p}$ in $L$}
		\STATE find $\overrightarrow{p}$ bounding hypercube 
		\STATE find $\overrightarrow{p}$ bounding simplex (permutation method)
		\STATE $\overrightarrow{\lambda} \leftarrow$ find $\overrightarrow{p}$ barycentric coefficients 
		\STATE $\overrightarrow{\hat{\lambda}} \leftarrow$ normalize($\overrightarrow{\lambda}$)
		\ENDFOR
		\STATE \textbf{Assign}
		\FORALL{$\overrightarrow{\phi}$ in $emb-set$}
		\STATE $inds \leftarrow$ find vertices from hypercube and simplex locations
		\FORALL{$i$ in $inds$}
		\STATE $\overrightarrow{\phi}_{(i)} \leftarrow \overrightarrow{\hat{\lambda}}(j(i))$ -- j is the assigning function between the coef. vector and embedding vector
		\ENDFOR
		\ENDFOR
		
		\RETURN $\overrightarrow{\phi}$

		
		\end{algorithmic}
	\end{algorithm}